\documentclass[a4paper]{article}

\usepackage{a4wide}
\usepackage{color}
\usepackage{graphicx}
\usepackage[version=3]{mhchem}
\usepackage{natbib}
\usepackage[hyperindex, backref, colorlinks,citecolor={red},linkcolor={blue}]{hyperref} % references for compilation with "latex" (-> dvi -> ps -> pdf)
\usepackage{textcomp}
\usepackage{longtable}
\usepackage{array}          %clean table lines
\usepackage{booktabs}  	    %\specialrule
\usepackage{wasysym}	
%\usepackage{here}
%\usepackage{letterspace}

\newcommand{\dt}[1]{\frac{d [#1]}{d t}}
\newcommand{\pd}[2]{\frac{\partial [#1]}{\partial [#2]}}

\newcommand{\co}{[\sum \ce{CO_2}]}
\newcommand{\nh}{[\sum \ce{NH_4^+}]}
\newcommand{\so}{[\sum \ce{HSO_4^-}]}
\newcommand{\hs}{[\sum \ce{H_2S}]}
\newcommand{\h}{[\ce{H^+}]}

\newcommand{\con}{\sum \ce{CO_2}}
\newcommand{\nhn}{\sum \ce{NH_4^+}}
\newcommand{\son}{\sum \ce{HSO_4^-}}
\newcommand{\hsn}{\sum \ce{H_2S}}
\newcommand{\hn}{\ce{H^+}}

\newcommand{\molin}{\frac{mol}{kg-solution}}
\newcommand{\molal}{\frac{mol}{kg-\ce{H_2O}}}
%\newcommand{\permill}{[\mbox{\textperthousand}]}
\newcommand{\permill}{[\mbox{\permil}]}
\newcommand{\T}{\frac{T}{K}}
\newcommand{\Tc}{\frac{T}{^\circ C}}
\newcommand{\Tca}{\left(\Tc\right)^2}
\newcommand{\Tcb}{\left(\Tc\right)^3}
\newcommand{\Tcc}{\left(\Tc\right)^4}
\newcommand{\Tcd}{\left(\Tc\right)^5}
\newcommand{\lnT}{\; \ln{\left(\frac{T}{K}\right)}}


% enumerate and add to table down to subparagraph
\addtocounter{tocdepth}{+3}
\addtocounter{secnumdepth}{+3}


\begin{document}

\title{\LARGE \textbf{\textsf{AquaEnv} - Constants and Formulae}}
\author{A.F. Hofmann}

\footnotetext[3]{corresponding author (a.hofmann@nioo.knaw.nl) }


\maketitle
\section{Constants and formulae} \label{app: consts}

\subsection{Chemical constants used in \textsf{AquaEnv}}

\subsubsection{Elements of list \texttt{PhysChemConst}}
\begin{scriptsize}
\begin{tabular}{|lrlp{4cm}p{6cm}|} \hline
\texttt{absZero} & -273.15     & \textdegree$ $C    & \citep{Dickson2007}    & absolute zero\\
\texttt{R}       & 83.14472    & (bar*cm3)/(mol*K)  & \citep{Dickson2007}    & ideal gas constant\\
\texttt{F}       & 96485.3399  & C/mol              & \citep{Dickson2007}    & Faraday constant\\
\texttt{e}       & 79          & -                  & \citep{Zeebe2001}      & relative dielectric constanf of seawater\\ \hline
\texttt{K\_HNO2} & 1.584893e-3 & mol/l              & \citep{Riordan2005}    & approximative dissociation constant of $\rm HNO_2$, NBS pH scale, hybrid constant\\
\texttt{K\_HNO3} & 23.44       & mol/kg-soln        & \citep{Boudreau1996, Soetaert2007} & approximative dissociation constant of $\rm HNO_3$,  assumed on mol/kg-soln and free pH scale, stoichiometric constant\\
\texttt{K\_H2SO4} & 100         & mol/kg-soln        & \citep{Atkins1996}     & approximative dissociation constant of $\rm H_2SO_4$, assumed on mol/kg-soln and free pH scale, stoichiometric constant\\
\texttt{K\_HS}    & 1.1e-12     & mol/kg-soln        & \citep{Atkins1996}     & approximative dissociation constant of $\rm HS$,  assumed on mol/kg-soln and free pH scale, stoichiometric constant\\ \hline
\end{tabular}
\end{scriptsize}


\subsubsection{Elements of list \texttt{MeanMolecularMass}}
The list  \texttt{MeanMolecularMass} contains mean molecular weights in g/mol. The list is taken from \citet[chap. 5, p. 3]{DOE1994} and \citet[chap. 5, p. 4]{Dickson2007}.\\
\begin{center}
\begin{tabular}{|ll|} \hline
\texttt{Cl}  & 35.453\\
\texttt{SO4} & (32.065+4$\;$(15.999))\\
\texttt{Br}  & 79.904\\
\texttt{F}   & 18.998\\
\texttt{Na}  & 22.990\\
\texttt{Mg}  & 24.3050\\
\texttt{Ca}  & 40.078\\
\texttt{K}   & 39.098\\
\texttt{Sr}  & 87.62\\
\texttt{B}   & 10.811\\ \hline
\end{tabular}
\end{center}

\subsubsection{Elements of list \texttt{ConcRelCl}}
The list \texttt{ConcRelCl} contains relative concentrations of key chemical species in seawater with respect to chlorinity (\citet[chap. 5, p. 11]{DOE1994} and \citet[chap. 5, p. 10]{Dickson2007})\\
\begin{center}
\begin{tabular}{|ll|} \hline
\texttt{Cl}  & 0.99889\\
\texttt{SO4} & 0.1400\\
\texttt{Br}  & 0.003473\\
\texttt{F}   & 0.000067\\
\texttt{Na}  & 0.55661\\
\texttt{Mg}  & 0.06626\\
\texttt{Ca}  & 0.02127\\
\texttt{K}   & 0.0206\\
\texttt{Sr}  & 0.00041\\
\texttt{B}   & 0.000232\\ \hline
\end{tabular}
\end{center}

\subsection{Chlorinity \texttt{Cl} as a function of salinity \texttt{S}}
Chlorinity \texttt{Cl} (in \permil) is calculated from salinity \texttt{S} using a relation given in \citet[chap. 5, p. 11]{DOE1994} and \citet[p. 100]{Zeebe2001}

\begin{equation}
\rm \mbox{\texttt{Cl}} = \frac{\mbox{\texttt{S}}}{1.80655}
\end{equation}

\subsection{Total concentrations of key chemical species in seawater as a function of chlorinity \texttt{Cl}}
As described in \citet[chap. 5, p. 11]{DOE1994} and \citet[chap. 5, p. 10]{Dickson2007}, values in lists \texttt{MeanMolecularMass} and \texttt{ConcRelCl} are used to calculate the total concentration [X] (in mol/kg-soln) of chemical species X in seawater\footnote{Note that the solution must have seawater composition, otherwise the relation given here is void.} according to the relation

\begin{equation}
\rm [X] = \frac{\mbox{\texttt{ConcRelCl\textdollar$ $X}}}{\mbox{\texttt{MeanMolecularMass\textdollar$ $X}}}\;\texttt{Cl}
\end{equation}

\subsection{Ionic strength \texttt{I} as function of salinity \texttt{S}}
According to \citet[chapter 5, p. 13, 15]{DOE1994},  \citet[ p.12]{Zeebe2001}, and \citet[ p.257]{Roy1993b}, I (in mol/kg-$\rm H_2O$) is calculated as
\begin{equation}
\rm \mbox{\texttt{I}} = \frac{19.924\;\mbox{\texttt{S}}}{1000-1.005\;\mbox{\texttt{S}}}
\end{equation}
Note that the approximation I/(mol/kg-solution) $\approx$ 0.0199201 \texttt{S} is given in \citet[  p. 428.]{Millero1982}.
This relationship converted into mol/kg-$\rm H_2O$ and the last digits adjusted (from 0.0199201 to 0.019924) results in the formula used here.\\

\subsection{Relation between water depth \texttt{d} and gauge pressure \texttt{p}}
Although the relation between gauge pressure \texttt{p} (total pressure minus atmospheric pressure, see \cite{Feistel2008})  and water depth \texttt{d} can be approximated by 
\begin{equation}
 \rm \mbox{\texttt{p}} = 0.1\;\mbox{\texttt{d}}\;1.01325
\end{equation}
since \texttt{p} increases per m of water depth \texttt{d} by approximately $\frac{1}{10}$ of 1 atm (= 1.01325 bar \citet[chap. 5, p. 3]{Dickson2007}),
here, the relation given by \cite{Fofonoff1983} as implemented in \cite{marelac} is used
\begin{equation}
\rm \mbox{\texttt{d}} = \frac{(9.72659 + (-2.2512\,10^{-5} + (2.279\,10^{-10} - 1.82\,10^{-15} \, \mbox{\texttt{p}}) \, \mbox{\texttt{p}})\, \mbox{\texttt{p}})\, \mbox{\texttt{p}}}{g + 1.092\,10^{-6} \, \mbox{\texttt{p}}}
\end{equation}
where \mbox{\texttt{p}} is the gauge pressure in dbar (deci-bar) and $\rm g$ the earth's gravity in m/$\rm s^2$. $\rm g$ is calculated from the latitude lat (in degrees, -90 to 90, if not given lat=0 is assumed) as given in \cite{Fofonoff1983}
and implemented in \cite{marelac}
\begin{equation}
\rm g = 9.780318\, (1+ ( 0.0052788 + 2.36\,10^{-5}\,\sin(lat\,\frac{\Pi}{180}))\,\sin(lat\,\frac{\Pi}{180}))
\end{equation}


\subsection{Seawater \texttt{density} as function of salinity \texttt{S} and temperature \texttt{t}}
According to \citep{Millero1981} as reprinted in \citet[chap. 5, p. 6f]{DOE1994} the density of seawater $\rho_{SeaWater}$ (in $\frac{kg}{m^3}$; \texttt{density} in an object of class \textit{aquaenv}) can be calculated as
%\begin{center}
\begin{eqnarray}
\rm \rho_{SeaWater} &=&\rm \rho_{Water} + A\;\mbox{\texttt{S}} + B\;\mbox{\texttt{S}}^{1.5} + C\; \mbox{\texttt{S}}^2\\
\rm A &=& \rm 0.824493 - 4.0899 \; 10^{-3} \, \mbox{\texttt{t}} + 7.6438 \; 10^{-5} \, \mbox{\texttt{t}}^2 - 8.2467 \; 10^{-7}  \,\mbox{\texttt{t}}^3\\
  && \rm + 5.3875 \; 10^{-9} \, \mbox{\texttt{t}}^4\\
\rm B &=& \rm -5.72466 \; 10^{-3} + 1.0227 \; 10^{-4} \, \mbox{\texttt{t}} - 1.6546\; 10^{-6} \, \mbox{\texttt{t}}^2 \\
\rm C &=& 4.8314\;10^{-4} \\
\rho_{Water} &=& 999.842594 + 6.793952\;10^{-2} \, \mbox{\texttt{t}} - 9.095290\;10^{-3} \, \mbox{\texttt{t}}^2\\
                                    && + 1.001685\;10^{-4}\,\mbox{\texttt{t}}^3 - 1.120083\;10^{-6} \, \mbox{\texttt{t}}^4 + 6.536332\;10^{-9} \, \mbox{\texttt{t}}^6
\end{eqnarray}
%\end{center}
\noindent
with \texttt{t} representing the temperature in \textdegree$ $C and $\rho_{Water}$ the density of fresh water in in kg/$\rm m^3$.

\subsection{Gas-exchange constants, dissociation constant, and solubility products as functions of salinity \texttt{S}, (absolute) temperature \texttt{T}, and gauge pressure \texttt{p}}

Empirical formulations for the temperature and salinity dependency of all gas exchange constants, equilibrium constants and solubility products calcuated in \textsf{AquaEnv} can be brought into the generic forms
\begin{equation}
\rm \ln{\frac{K_X}{k_0^\circ}} = A + \frac{B}{\mbox{\texttt{T}}} + C \ln(\mbox{\texttt{T}}) + D \; \mbox{\texttt{T}} + E \;\mbox{\texttt{T}}^2
\end{equation}
or
\begin{equation}
\rm \log_{10}{\frac{K_X}{k_0^\circ}} = A' + \frac{B'}{\mbox{\texttt{T}}} + C' \log_{10}(\mbox{\texttt{T}}) + D' \; \mbox{\texttt{T}} + E' \;\mbox{\texttt{T}}^2
\end{equation}
or
\begin{equation}
\rm \log_{10}{\frac{K_X}{k_0^\circ}} = A'' + \frac{B''}{\mbox{\texttt{T}}} + C'' \ln(\mbox{\texttt{T}}) + D'' \; \mbox{\texttt{T}} + E'' \;\mbox{\texttt{T}}^2 \label{eq: logk}
\end{equation}
with \texttt{T} being the temperature in Kelvin, \texttt{S} the salinity, $k_0^\circ$ the concentration unit of the constant, and A, B, C, D, E, and the respective variables with a prime (') being functions of salinity \texttt{S}.
In the following we will give A, B, C, D, and E,  or A', B', C', D', and E', or A'', B'', C'', D'', and E'' for each calculated constant.

\subsubsection{Gas-exchange constants (Henry's constants) as functions of salinity \texttt{S} and temperature \texttt{T}}

The following table shows the coefficients for gas exchange constants in \textsf{AquaEnv}, with $f\ce{CO_2}$ being the fugacity of $\ce{CO_2}$.

\clearpage

\begin{longtable}{|p{.6\textwidth}|p{0.07\textwidth}cp{0.2\textwidth}|}\specialrule{1pt}{0pt}{0pt}
\multicolumn{4}{|l|}{\texttt{K0\_CO2} \textbf{: solubility of $\ce{CO_2}$ in seawater}}\\ \specialrule{1pt}{0pt}{0pt}
A $= 0.023517 \mbox{\texttt{S}} - 167.81077$& \texttt{CO2\_sat} &=& $f\ce{CO_2} \; $\texttt{K0\_CO2}\\
B $= 9345.17$&&&\\
C $= 23.3585$& ${k_0^\circ}$ &=& $\left[\frac{mol}{kg-solution \; atm}\right]$\\
D $= -2.3656 \; 10^{-4} \mbox{\texttt{S}}$&&&\\
E $= 4.7036 \; 10^{-7} \mbox{\texttt{S}}$&&&\\ \hline
\multicolumn{4}{|l|}{\textit{References:} \citet{Weiss1974} (original), \citet[chap. 5, p. 13]{DOE1994}, \citet[p. 663]{Millero1995},}\\
\multicolumn{4}{|l|}{\color{white} \textit{References:} \color{black} \citet[p. 257]{Zeebe2001}, and \citet[chap. 5, p. 12]{Dickson2007}} \\ \hline \specialrule{1pt}{2pt}{0pt}
\multicolumn{4}{|l|}{\texttt{K0\_O2} \textbf{: solubility of $\ce{O_2}$ in seawater} (\textbf{micro}mol per kg-soln and atm)}\\ \specialrule{1pt}{0pt}{0pt}
A $= -846.9978 - 0.037362 \; \mbox{\texttt{S}}$ & \texttt{O2\_sat} &=& $f\ce{O_2} \; $\texttt{K0\_O2}\ \\
B $=  25559.07 $ &&&\\
C $=  146.4813$ & ${k_0^\circ}$&=& $\left[\frac{\mu mol}{kg-solution \; atm}\right]$\\
D $= -0.22204 + 0.00016504 \; \mbox{\texttt{S}}$ &&&\\
E $= -2.0564 \; 10^{-7} \; \mbox{\texttt{S}}$ &&&\\ \hline
\multicolumn{4}{|l|}{\textit{References:} derived from \citet{Weiss1970}, agrees with data in \citet{Murray1969a}} \\ \hline
\end{longtable}
\noindent
Note that the formulation for \texttt{K0\_O2} has been derived using the formulation for a gravimetric $[\ce{O_2}]_{sat}$ given in \citet[Weiss, 1970]{Weiss1970}. It has been converted from ml-$\rm O_2$/kg-soln to $\mu$mol-$\rm O_2$/kg-soln using the molar volume of $\rm O_2$ calculated with the virial equation using a first virial coefficient for oxygen at 273.15 Kelvin of $\rm -22 \; cm^3/mol$ \citet{Atkins1996}, a value of $8.314472$ Nm/(Kelvin mol) for the gas constant R and an ambient pressure of 101325 N/$\rm m^2$. The expression for the Henry's constant has then been created by dividing the expression for the saturation concentration by $\rm f\ce{O_2} = 0.20946 \;atm$ \citep{Williams2004}.


\subsubsection{Stoichiometric acid base dissociation constants as functions of salinity \texttt{S} and temperature \texttt{T}}
The following table gives the coefficients of stoichiometric acid base dissociation constants in \textsf{AquaEnv}.  Note that if some of the coefficients A to E are not listed, they are to be considered zero. Note also that given references somtimes contain the formulae in different units or on different pH scales. The formulae provided in this table yield the dissociation constants on different pH scales and concentration units. In \textsf{AquaEnv}, constants that are not already on the free pH scale and in mol/kg-soln are converted to the free pH scale and mol/kg-soln.

%\newpage

\begin{longtable}{|p{.7\textwidth}|p{0.06\textwidth}cp{0.15\textwidth}|}\specialrule{1pt}{0pt}{0pt}
\multicolumn{3}{|l}{\textbf{\texttt{K\_HSO4} : $\ce{HSO_4^-} \rightleftharpoons \ce{H^+  +  SO_4^{2-}}$} ("dickson") } & \textbf{free }pH scale\\ \specialrule{1pt}{0pt}{0pt}
A = $324.57 \; \sqrt{\left(\frac{\mbox{\texttt{I}}}{m^\circ}\right)} - 771.54 \; \frac{\mbox{\texttt{I}}}{m^\circ} + 141.328 $ &\texttt{K\_HSO4}&=& $\frac{[\ce{H^+}]_F \; [\ce{SO_4^{2-}}]}{[\ce{HSO_4^-}]}$ \\
B = $35474 \;\frac{\mbox{\texttt{I}}}{m^\circ} + 1776 \; {\left(\frac{\mbox{\texttt{I}}}{m^\circ}\right)}^2  - 13856 \; \sqrt{\left(\frac{\mbox{\texttt{I}}}{m^\circ}\right)} - 2698 {\left(\frac{\mbox{\texttt{I}}}{m^\circ}\right)}^\frac{3}{2} - 4276.1$ &$k^\circ$ &=& $\molal$\\
C = $114.723 \; \frac{\mbox{\texttt{I}}}{m^\circ} - 47.986 \sqrt{\left(\frac{\mbox{\texttt{I}}}{m^\circ}\right)} - 23.093$&$m^\circ$ &=& $\molal$ \\ \hline
\multicolumn{4}{|l|}{\textit{References:} \citet[c. 5, p. 13]{DOE1994}, \citet[p. 260]{Zeebe2001}, \citet{Dickson1990a} (original)} \\ \hline
\specialrule{1pt}{2pt}{0pt}
\multicolumn{3}{|l}{\textbf{\texttt{K\_HSO4} : $\ce{HSO_4^-} \rightleftharpoons \ce{H^+  +  SO_4^{2-}}$} ("khoo")} & \textbf{free }pH scale\\ \specialrule{1pt}{0pt}{0pt}
A = $6.3451 + 0.5208\; \sqrt{\left(\frac{\mbox{\texttt{I}}}{m^\circ}\right)}$ &\texttt{K\_HSO4}&=& $\frac{[\ce{H^+}]_F \; [\ce{SO_4^{2-}}]}{[\ce{HSO_4^-}]}$ \\
B =  -647.59 &$k^\circ$ &=& $\molal$ \\
D = -0.019085 &$m^\circ$ &=& $\molal$ \\ \hline
\multicolumn{4}{|l|}{\textit{References:} \cite{Khoo1977} (original), \cite{Roy1993a}, \cite{Millero1995}, \cite{Lewis1998}} \\ \hline
\pagebreak
\specialrule{1pt}{2pt}{0pt}
\multicolumn{3}{|l}{\textbf{\texttt{K\_HF}: $\ce{HF} \rightleftharpoons \ce{H^+ + F^-}$} ("dickson") } & \textbf{free} pH scale\\ \specialrule{1pt}{0pt}{0pt}
A =  $1.525 \; \sqrt{\frac{\mbox{\texttt{I}}}{m^\circ}} - 12.641$ & \texttt{K\_HF} &=& $\frac{\ce{[H^+]}_F \; [\ce{F^-}]}{[\ce{HF}]}$\\
B = $1590.2$ &$k^\circ$ &=& $m^\circ$ = $\molal$ \\ \hline
\multicolumn{4}{|l|}{\textit{References:} \citet[p. 91]{Dickson1979} (origninal), \citet[p. 1783]{Dickson1987},}\\
\multicolumn{4}{|l|}{\color{white}\textit{References:} \color{black} \citet[p. 257]{Roy1993a}, \citet[c. 5, p. 15]{DOE1994}, \citet[p. 664]{Millero1995},}\\
\multicolumn{4}{|l|}{\color{white}\textit{References:} \color{black} \citet[p. 260]{Zeebe2001}}\\ \hline 
%\pagebreak
\specialrule{1pt}{2pt}{0pt}
\multicolumn{3}{|l}{\textbf{\texttt{K\_HF}: $\ce{HF} \rightleftharpoons \ce{H^+ + F^-}$} ("perez")}  & \textbf{total} pH scale\\ \specialrule{1pt}{0pt}{0pt}
A = $-9.68 + 0.111 \, \sqrt{S}$ & \texttt{K\_HF} &=& $\frac{\ce{[H^+]}_F \; [\ce{F^-}]}{[\ce{HF}]}$\\
B = $874$ &$k^\circ$ &=& $\molin$ \\ \hline
\multicolumn{4}{|l|}{\textit{References:} \citet[p. 91]{Perez1987a} (origninal), \citet[chap. 5, p. 14]{Dickson2007}}\\ \hline \specialrule{1pt}{2pt}{0pt}
\multicolumn{3}{|l}{\textbf{\texttt{K\_CO2}: $\ce{CO_2}(aq) + \ce{H_2O} \; (\rightleftharpoons \ce{H_2CO_3}) \; \rightleftharpoons \ce{H^+ + HCO_3^-}$} ("roy"; high salinities: $\mbox{\texttt{S}} > 5$)} & \textbf{total} pH scale\\ \specialrule{1pt}{0pt}{0pt}
A = $2.83655 -0.20760841\; \sqrt{\mbox{\texttt{S}}} + 0.08468345 \; \mbox{\texttt{S}} - 0.00654208 \; \mbox{\texttt{S}}^{\frac{3}{2}}$ & \texttt{K\_CO2} &=& $\frac{[\ce{H^+}] \; [\ce{HCO_3^-}]}{[\ce{CO_2(aq)}]}$\\
B = $- 2307.1266 - 4.0484 \; \sqrt{\mbox{\texttt{S}}}$ & $k^\circ$ &=& $\molal$\\
C = $- 1.5529413$ &&&\\ \hline
\multicolumn{4}{|l|}{\textit{References:} \citet[p. 254]{Roy1993a} (original), \citet[c. 5, p.14]{DOE1994}, \citet[p. 664]{Millero1995},}\\
\multicolumn{4}{|l|}{\color{white}\textit{References:} \color{black} \citet[p. 255]{Zeebe2001}}\\ \hline 
%\pagebreak 
\specialrule{1pt}{2pt}{0pt}
\multicolumn{3}{|l}{\textbf{\texttt{K\_CO2}: $\ce{CO_2}(aq) + \ce{H_2O} \; (\rightleftharpoons \ce{H_2CO_3}) \; \rightleftharpoons \ce{H^+ + HCO_3^-}$}("roy"; low salinities: $\mbox{\texttt{S}} \leq 5$)}  &\textbf{total} pH scale\\ \specialrule{1pt}{0pt}{0pt}
\multicolumn{4}{|l|}{A = $290.9097 - 228.39774\; \sqrt{\mbox{\texttt{S}}} +  54.20871 \; \mbox{\texttt{S}} - 3.969101\; \mbox{\texttt{S}}^{\frac{3}{2}}- 0.00258768 \; \mbox{\texttt{S}}^2$}\\
B = $-14554.21 + 9714.36839\; \sqrt{\mbox{\texttt{S}}} - 2310.48919 \; \mbox{\texttt{S}}+ 170.22169\; \mbox{\texttt{S}}^{\frac{3}{2}}$& \texttt{K\_CO2} &=& $\frac{[\ce{H^+}] \; [\ce{HCO_3^-}]}{[\ce{CO_2}(aq)]}$ \\
C = $- 45.0575+ 34.485796 \; \sqrt{\mbox{\texttt{S}}}- 8.19515\; \mbox{\texttt{S}}+ 0.60367\; \mbox{\texttt{S}}^{\frac{3}{2}}$ & $k^\circ$ &=& $\molal$\\ \hline
\multicolumn{4}{|l|}{\textit{References:} \citet[p. 256]{Roy1993a} (original, based on a temperature dependency restated in}\\ \multicolumn{4}{|l|}{\color{white}\textit{References:} \color{black} \citet{Millero1979}, originally given in \citet{Harned1943}. Note that there is a typesetting}\\
\multicolumn{4}{|l|}{\color{white}\textit{References:} \color{black}  error in \citet{Roy1993a}: The third value for $B$ is 2310.48919, not 310.48919)}\\
\multicolumn{4}{|l|}{\color{white}\textit{References:} \color{black}  \citet[p. 664]{Millero1995} (the typesetting error is corrected here. also, here it is mentioned that} \\
\multicolumn{4}{|l|}{\color{white}\textit{References:} \color{black} this formula should be used for $\mbox{\texttt{S}} \leq 5$. Note that both functions do not always intersect at }\\
\multicolumn{4}{|l|}{\color{white}\textit{References:} \color{black}  \texttt{S}=5. The true intersection is a function of \texttt{t}, is calculated in \textsf{AquaEnv}, and is used to }\\
\multicolumn{4}{|l|}{\color{white}\textit{References:} \color{black} decide which formula to use.)}\\ \hline
%\pagebreak 
\specialrule{1pt}{2pt}{0pt}
\multicolumn{3}{|l}{\textbf{\texttt{K\_CO2}: $\ce{CO_2}(aq) + \ce{H_2O} \; (\rightleftharpoons \ce{H_2CO_3}) \; \rightleftharpoons \ce{H^+ + HCO_3^-}$} ("lueker")} & \textbf{total} pH scale\\ \specialrule{1pt}{0pt}{0pt}
A'' = $61.2172 + 0.011555\; \mbox{\texttt{S}} - 0.0001152 \; \mbox{\texttt{S}}^2$ & \texttt{K\_CO2} &=& $\frac{[\ce{H^+}] \; [\ce{HCO_3^-}]}{[\ce{CO_2(aq)}]}$\\
B'' = $- 3633.86$ & $k^\circ$ &=& $\molin$\\
C'' = $- 9.67770$ &&&\\ \hline
\multicolumn{4}{|l|}{\textit{References:} \citet{Lueker2000} (original), \citet[chap. 5, p.13-14]{Dickson2007}}\\ \hline \specialrule{1pt}{2pt}{0pt}
\multicolumn{3}{|l}{\textbf{\texttt{K\_CO2}: $\ce{CO_2}(aq) + \ce{H_2O} \; (\rightleftharpoons \ce{H_2CO_3}) \; \rightleftharpoons \ce{H^+ + HCO_3^-}$} ("millero")} & \textbf{SW} pH scale\\ \specialrule{1pt}{0pt}{0pt}
A'' = $126.34048 - 0.0331\;\mbox{\texttt{S}} + 0.0000533 \; \mbox{\texttt{S}}^2 - 13.4191 \; \sqrt{\mbox{\texttt{S}}}$& \texttt{K\_CO2} &=& $\frac{[\ce{H^+}] \; [\ce{HCO_3^-}]}{[\ce{CO_2(aq)}]}$\\
B'' = $-6320.813 +  6.103\;\mbox{\texttt{S}} + 530.123\; \sqrt{\mbox{\texttt{S}}}$ & $k^\circ$ &=& $\molin$\\
C'' = $-19.568224 + 2.06950\; \sqrt{\mbox{\texttt{S}}}$ &&&\\ \hline
\multicolumn{4}{|l|}{\textit{References:} \citet{Millero2006} (original)} \\ \hline \specialrule{1pt}{2pt}{0pt}
\multicolumn{3}{|l}{\textbf{\texttt{K\_HCO3}: $\ce{HCO_3^-} \rightleftharpoons \ce{H^+ + CO_3^{2-}}$} ("roy"; high salinities: $\mbox{\texttt{S}} > 5$) }& \textbf{total} pH scale\\ \specialrule{1pt}{0pt}{0pt}
A = $-9.226508 -0.106901773\; \sqrt{\mbox{\texttt{S}}} + 0.1130822  \; \mbox{\texttt{S}}  - 0.00846934\; \mbox{\texttt{S}}^{\frac{3}{2}}$ & \texttt{K\_HCO3} &=& $\frac{[\ce{H^+}] \; [\ce{CO_3^{2-}}]}{[\ce{HCO_3^-}]}$\\
B = $-3351.6106 - 23.9722 \; \sqrt{\mbox{\texttt{S}}}$ & $k^\circ$ &=& $\molal$\\
C = $- 0.2005743$ &&&\\ \hline
\multicolumn{4}{|l|}{\textit{References:} \citet[p. 254]{Roy1993a} (original), \citet[c. 25, p.15]{DOE1994}, \citet[p. 664]{Millero1995},}\\
 \multicolumn{4}{|l|}{\color{white}\textit{References:} \color{black} \citet[p. 255]{Zeebe2001} }\\ \hline  
\pagebreak
\specialrule{1pt}{2pt}{0pt}
\multicolumn{3}{|l}{\textbf{\texttt{K\_HCO3}: $\ce{HCO_3^-} \rightleftharpoons \ce{H^+ + CO_3^{2-}}$} ("roy"; low salinities: $\mbox{\texttt{S}} \leq 5$) }& \textbf{total} pH scale\\ \specialrule{1pt}{0pt}{0pt}
\multicolumn{4}{|l|}{A = $207.6548 -167.69908\; \sqrt{\mbox{\texttt{S}}} +    39.75854\; \mbox{\texttt{S}}  -2.892532\; \mbox{\texttt{S}}^{\frac{3}{2}} - 0.00613142\; \mbox{\texttt{S}}^2$}\\
B =  $-11843.79 + 6551.35253\; \sqrt{\mbox{\texttt{S}}} - 1566.13883 \; \mbox{\texttt{S}}+ 116.270079\; \mbox{\texttt{S}}^{\frac{3}{2}}$& \texttt{K\_HCO3} &=& $\frac{[\ce{H^+}] \; [\ce{CO_3^{2-}}]}{[\ce{HCO_3^-}]}$\\
C =  $- 33.6485 + 25.928788\; \sqrt{\mbox{\texttt{S}}} - 6.171951\; \mbox{\texttt{S}} +  0.45788501\; \mbox{\texttt{S}}^{\frac{3}{2}}$& $k^\circ$ &=& $\molal$\\ \hline
\multicolumn{4}{|l|}{\textit{References:} \citet[p. 256]{Roy1993a} (original, based on a temperature depencence restated in}\\ \multicolumn{4}{|l|}{\color{white}\textit{References:} \color{black} \citet{Millero1979}, originally given in \citet{Harned1941}), \citet[p. 664]{Millero1995}}\\
\multicolumn{4}{|l|}{\color{white}\textit{References:} \color{black} (here it is mentioned that this formula should be used for $\mbox{\texttt{S}} \leq 5$. Note that both functions}\\
\multicolumn{4}{|l|}{\color{white}\textit{References:} \color{black} do not always intersect at \texttt{S}=5. The true intersection is a function of \texttt{t}, is calculated in}\\
\multicolumn{4}{|l|}{\color{white}\textit{References:} \color{black} \textsf{AquaEnv}, and is used to decide which formula to use.)}\\ \hline 
%\pagebreak
\specialrule{1pt}{2pt}{0pt}
\multicolumn{3}{|l}{\textbf{\texttt{K\_HCO3}: $\ce{HCO_3^-} \rightleftharpoons \ce{H^+ + CO_3^{2-}}$} ("lueker")} & \textbf{total} pH scale\\ \specialrule{1pt}{0pt}{0pt}
A'' =  $-25.9290 + 0.01781\; \mbox{\texttt{S}} - 0.0001122\; \mbox{\texttt{S}}^2$ & \texttt{K\_HCO3}&=& $\frac{[\ce{H^+}] \; [\ce{CO_3^{2-}}]}{[\ce{HCO_3^-}]}$\\
B'' = $- 471.78$ & $k^\circ$ &=& $\molin$\\
C'' = $3.16967$ &&&\\ \hline
\multicolumn{4}{|l|}{\textit{References:} \citet{Lueker2000} (original), \citet[chap. 5, p.14]{Dickson2007}}\\ \hline 
%\pagebreak
\specialrule{1pt}{2pt}{0pt}
\multicolumn{3}{|l}{\textbf{\texttt{K\_HCO3}: $\ce{HCO_3^-} \rightleftharpoons \ce{H^+ + CO_3^{2-}}$} ("millero")} & \textbf{SW} pH scale\\ \specialrule{1pt}{0pt}{0pt}
A'' =  $90.18333  - 0.1248\;\mbox{\texttt{S}} + 0.0003687\; \mbox{\texttt{S}}^2 - 21.0894\; \sqrt{\mbox{\texttt{S}}}$ & \texttt{K\_HCO3}&=& $\frac{[\ce{H^+}] \; [\ce{CO_3^{2-}}]}{[\ce{HCO_3^-}]}$\\
B'' = $-5143.692 + 20.051\;\mbox{\texttt{S}} + 772.483\; \sqrt{\mbox{\texttt{S}}}$ & $k^\circ$ &=& $\molin$\\
C'' = $-14.613358 +  3.3336\; \sqrt{\mbox{\texttt{S}}}$ &&&\\ \hline
\multicolumn{4}{|l|}{\textit{References:} \citet{Millero2006} (original)}\\ \hline 
\specialrule{1pt}{2pt}{0pt}
\multicolumn{3}{|l}{\textbf{\texttt{K\_W}: $\ce{H_2O} \rightleftharpoons \ce{H^+ + OH^-}$}} & \textbf{total} pH scale\\ \specialrule{1pt}{0pt}{0pt}
A = $148.9652 - 5.977\; \sqrt{\mbox{\texttt{S}}}  - 0.01615 \; \mbox{\texttt{S}}$ &\texttt{K\_W} &=& $[\ce{H^+}] \; [\ce{OH^-}]$\\
B = $- 13847.26 + 118.67\; \sqrt{\mbox{\texttt{S}}}$ & $k^\circ$ &=& ${\left(\frac{\rm mol}{\rm kg-solution}\right)}^2$\\
C = $- 23.6521+ 1.0495 \; \sqrt{\mbox{\texttt{S}}}$ &&&\\ \hline
\multicolumn{4}{|l|}{\textit{References:} \citet[p.670]{Millero1995} (original), \citet[c. 5, p. 18]{DOE1994} (update 1997 cites \citet{Millero1995}),}\\
\multicolumn{4}{|l|}{\color{white}\textit{References:} \color{black} \citet[p. 258]{Zeebe2001}, \citet[chap. 5, p.16]{Dickson2007}} \\ \hline %\pagebreak 
\specialrule{1pt}{2pt}{0pt}
\multicolumn{3}{|l}{\textbf{\texttt{K\_BOH3}: $\ce{B(OH)_3} \rightleftharpoons \ce{H^+ + B(OH)_4^-}$}} & \textbf{total} pH scale\\ \specialrule{1pt}{0pt}{0pt}
A = $ 148.0248 + 137.1942 \; \sqrt{\mbox{\texttt{S}}} + 1.62142 \mbox{\texttt{S}}$& \texttt{K\_BOH3} &=&  $\frac{[\ce{H^+}] \; [\ce{B(OH)_4^-}]}{[\ce{B(OH)_3}]}$\\
B = $-8966.90 - 2890.53 \; \sqrt{\mbox{\texttt{S}}} - 77.942 \mbox{\texttt{S}} + 1.728 \; \mbox{\texttt{S}}^{\frac{3}{2}} -0.0996 \; \mbox{\texttt{S}}^2$ & $k^\circ$ &=& $\molin$\\
C = $-24.4344 - 25.085 \; \sqrt{\mbox{\texttt{S}}} - 0.2474 \; \mbox{\texttt{S}}$&&&\\
D = $0.053105 \; \sqrt{\mbox{\texttt{S}}}$ &&&\\ \hline
\multicolumn{4}{|l|}{\textit{References:} \citet[p. 763]{Dickson1990} (or.), \citet[c. 5, p. 14]{DOE1994}, \citet[p. 669]{Millero1995},}\\
\multicolumn{4}{|l|}{\color{white}\textit{References:} \color{black} \citet[p. 262]{Zeebe2001} , agrees with data in \citet{Roy1993}} \\ \hline
%\pagebreak
 \specialrule{1pt}{2pt}{0pt}
\multicolumn{3}{|l}{\textbf{\texttt{K\_NH4}: $\ce{NH_4^+} \rightleftharpoons \ce{H^+ + NH_3}$}} & \textbf{SW} pH scale\\ \specialrule{1pt}{0pt}{0pt}
A = $-0.25444 + 0.46532 \; \sqrt{\mbox{\texttt{S}}} -0.01992 \; \mbox{\texttt{S}}$ & \texttt{K\_NH4} &=&  $\frac{[\ce{H^+}] \; [\ce{NH_3}]}{[\ce{NH_4^+}]}$ \\
B = $-6285.33 - 123.7184\; \sqrt{\mbox{\texttt{S}}} + 3.17556 \; \mbox{\texttt{S}}$ &$k^\circ$ &=& $\molin$\\
D = $0.0001635$ &&&\\ \hline
\multicolumn{4}{|l|}{\textit{References:} \citet[p. 671]{Millero1995}, \citet{Millero1995a} (original),}\\
\multicolumn{4}{|l|}{\color{white}\textit{References:} \color{black} corrections of \citet{Millero1995} in \citet{Lewis1998} give pH scale } \\ \hline
%\pagebreak
\specialrule{1pt}{2pt}{0pt}
 \multicolumn{3}{|l}{\textbf{\texttt{K\_H2S}: $\ce{H_2S} \rightleftharpoons \ce{H^+ + HS^-}$}} & \textbf{total} pH scale\\ \specialrule{1pt}{0pt}{0pt}
 A = $225.838 + 0.3449 \; \sqrt{\mbox{\texttt{S}}} - 0.0274 \; \mbox{\texttt{S}}$ & \texttt{K\_H2S} &=& $\frac{\ce{[H^+]} \; [\ce{HS^-}]}{[\ce{H_2S}]}$\\
 B = $- 13275.3$ & $k^\circ$ &=& $\molin$  \\
 C = $- 34.6435$&&&\\ \hline
 \multicolumn{4}{|l|}{\textit{References:} \citet[p. 671]{Millero1995}, \citet{Millero1988} (original),}\\
  \multicolumn{4}{|l|}{\color{white}\textit{References:} \color{black} corrections of \citet{Millero1995} in \citet{Lewis1998} give pH scale} \\ \hline
\pagebreak
 \specialrule{1pt}{2pt}{0pt}
 \multicolumn{3}{|l}{\textbf{\texttt{K\_H3PO4}: $\rm H_3PO_4 \rightleftharpoons H^+ + H_2PO_4^-$}} & \textbf{total} pH scale\\ \specialrule{1pt}{0pt}{0pt}
 A = $115.525 + 0.69171 \; \sqrt{\mbox{\texttt{S}}}  - 0.01844\; \mbox{\texttt{S}}$ & \texttt{K\_H3PO4} &=& $\rm \frac{[H^+] \; [H_2PO_4^-]}{[H_3PO_4]}$\\
 B = $- 4576.752 -106.736 \; \sqrt{\mbox{\texttt{S}}}- 0.65643\; \mbox{\texttt{S}}$ & $k^\circ$ &=& $\molin$ \\
 C = $- 18.453 $ &&&\\ \hline
 \multicolumn{4}{|l|}{\textit{References:} \citet[chap. 5, p 16]{DOE1994}, \citet[p.670]{Millero1995}, (original) }\\
  \multicolumn{4}{|l|}{\color{white}\textit{References:} \color{black} \citet[chap. 5, p.15]{Dickson2007} agrees with data in \citet{Dickson1979a}} \\ \hline \specialrule{1pt}{2pt}{0pt}
 \multicolumn{3}{|l}{\textbf{\texttt{K\_H2PO4} : $\rm H_2PO_4^- \rightleftharpoons H^+ + HPO_4^{2-}$}} & \textbf{total} pH scale\\ \specialrule{1pt}{0pt}{0pt}
A = $172.0883 +  1.3566\; \sqrt{\mbox{\texttt{S}}} - 0.05778 \; \mbox{\texttt{S}}$ & \texttt{K\_H2PO4} &=& $\rm \frac{[H^+] \; [HPO_4^{2-}]}{[H_2PO_4^-]}$\\
 $B = - 8814.715 - 160.340 \; \sqrt{\mbox{\texttt{S}}} +  0.37335\; \mbox{\texttt{S}}$ & $k^\circ$ &=& $\molin$\\
 $C = - 27.927 $ &&&\\ \hline
 \multicolumn{4}{|l|}{\textit{References:} \citet[chap. 5, p 16]{DOE1994}, \citet[p.670]{Millero1995} (original), }\\
  \multicolumn{4}{|l|}{\color{white}\textit{References:} \color{black} \citet[chap. 5, p.15]{Dickson2007}, agrees with data in \citet{Dickson1979a}}\\ \hline 
%\pagebreak
\specialrule{1pt}{2pt}{0pt}
 \multicolumn{3}{|l}{\textbf{\texttt{K\_HPO4} : $\rm HPO_4^{2-} \rightleftharpoons H^+ + PO_4^{3-}$}} & \textbf{total} pH scale\\ \specialrule{1pt}{0pt}{0pt}
A = $ - 18.141  + 2.81197 \; \sqrt{\mbox{\texttt{S}}} -  0.09984\; \mbox{\texttt{S}}$ & \texttt{K\_HPO4} &=& $\rm \frac{[H^+] \; [PO_4^{3-}]}{[HPO_4^{2-}]}$\\
B = $-  3070.75 + 17.27039\; \sqrt{\mbox{\texttt{S}}} -  44.99486 \; \mbox{\texttt{S}}$ & $k^\circ$ &=& $\molin$\\ \hline
 \multicolumn{4}{|l|}{\textit{References:} \citet[chap. 5, p 17]{DOE1994}, \citet[p.670]{Millero1995} (original),}\\
  \multicolumn{4}{|l|}{\color{white}\textit{References:} \color{black}  \citet[chap. 5, p.15]{Dickson2007}, agrees with data in \citet{Dickson1979a}}\\ \hline 
%\pagebreak
\specialrule{1pt}{2pt}{0pt}
\multicolumn{3}{|l}{\textbf{\texttt{K\_SiOH4}: $\rm Si(OH)_4 \rightleftharpoons H^+ + SiO(OH)_3^-$}} & \textbf{total} pH scale\\ \specialrule{1pt}{0pt}{0pt}
 A = $117.385+ 3.5913\; \sqrt{\frac{\mbox{\texttt{I}}}{m^\circ}}- 1.5998\;\frac{\mbox{\texttt{I}}}{m^\circ}+ 0.07871 \; \left(\frac{\mbox{\texttt{I}}}{m^\circ}\right)^2$& \texttt{K\_SiOH4} &=&  $\rm \frac{[H^+] \; [SiO(OH)_3^-]}{[Si(OH)_4]}$\\
 B = $-8904.2-458.79\; \sqrt{\frac{\mbox{\texttt{I}}}{m^\circ}} + 188.74\; \frac{\mbox{\texttt{I}}}{m^\circ} - 12.1652 \left(\frac{\mbox{\texttt{I}}}{m^\circ}\right)^2$ & $k^\circ$ &=& $\molal$ \\
 C = $-19.334$ & $m^\circ$ &=& $\molal$\\ \hline
 \multicolumn{4}{|l|}{\textit{References:}  \citet{Millero1988} (original), \citet[chapter 5, p 17]{DOE1994}, \citet[p.671]{Millero1995}} \\ \hline %\pagebreak 
\specialrule{1pt}{2pt}{0pt}
 \multicolumn{3}{|l}{\textbf{\texttt{K\_SiOOH3}: $\rm SiO(OH)_3^- \rightleftharpoons H^+ + SiO_2(OH)_2^{2-}$}} & \textbf{total} pH scale\\ \specialrule{1pt}{0pt}{0pt}
 A = $8.96$& \texttt{K\_SiOOH3}  &=&  $\rm \frac{[H^+] \; [SiO_2(OH)_2^{2-}]}{[SiO(OH)_3^-]}$\\
 B = $-4465.18$ & $k^\circ$ &=& $\molal$ \\
 D = $0.021952$ & &&\\ \hline
 \multicolumn{4}{|l|}{\textit{References:}  \citet{Wischmeyer2003} (original; including corrections by co-author D. Wolf-Gladrow)} \\ \hline
\end{longtable}


\subsection{Stoichiometric solubility products as functions of salinity \texttt{S} and temperature \texttt{T}}
The following table shows the coefficients for the stoichiometric solubility products for calcite and aragonite in \textsf{AquaEnv}.
%\newpage
\begin{longtable}{|p{.6\textwidth}|p{0.07\textwidth}cp{0.2\textwidth}|}\specialrule{1pt}{0pt}{0pt}
\multicolumn{4}{|l|}{\texttt{Ksp\_calcite} \textbf{: solubility product of calcite
}}\\ \specialrule{1pt}{0pt}{0pt}
$A' = -171.9065 -   0.77712\;\sqrt{\texttt{S}} - 0.07711\;S + 0.0041249\;S^{1.5}$& \texttt{Ksp\_cal} &=& $[\rm CO_3^{2-}] \; [Ca^{2+}]$\\
$B' = 2839.319  +    178.34\;\sqrt{\texttt{S}}$&&&\\
$C' = 71.595$& ${k_0^\circ}$ &=& $\left[\left(\frac{mol}{kg-solution}\right)^2\right]$\\
$D' = -0.077993 + 0.0028426\;\sqrt{\texttt{S}}$&&&\\ \hline
\multicolumn{4}{|l|}{\textit{References:} \citet{Mucci1983} (original), \citet[p. 160]{Boudreau1996},}\\
\multicolumn{4}{|l|}{\color{white} \textit{References:} \color{black} (note that the second value for $A'$ is -0.77712 not -0.7712 as cited in \citet{Boudreau1996})} \\ \hline 
\pagebreak
\specialrule{1pt}{2pt}{0pt}
\multicolumn{4}{|l|}{\texttt{Ksp\_aragonite} \textbf{: solubility product of aragonite
}}\\ \specialrule{1pt}{0pt}{0pt}
$A' = -171.945  -  0.068393\;\sqrt{\texttt{S}} - 0.10018\;S + 0.0059415\;S^{1.5}$ & \texttt{Ksp\_ara} &=& $[\rm CO_3^{2-}] \; [Ca^{2+}]$\\
$B' = 2903.293  +    88.135\;\sqrt{\texttt{S}}$&&&\\
$C' = 71.595$& ${k_0^\circ}$ &=& $\left[\left(\frac{mol}{kg-solution}\right)^2\right]$\\
$D' = -0.077993 + 0.0017276\;\sqrt{\texttt{S}}$&&&\\ \hline
\multicolumn{4}{|l|}{\textit{References:} \citet{Mucci1983} (original), \citet[p. 160]{Boudreau1996},}\\
\multicolumn{4}{|l|}{\color{white} \textit{References:} \color{black} (note that the second value for $D'$ is 0.0017276 not 0.001727 as cited in \citet{Boudreau1996})} \\ \hline
\end{longtable}

\subsection{Pressure correction of dissociation constants and solubility products}
Pressure has an effect on the stoichiometric acid-base dissociation constants  and the stoichiometric solubility products given in the  previous sections. As described in \citet[p. 675]{Millero1995} using corrections and assumptions from \citet[p. A-7]{Lewis1998} the effect of pressure can be accounted for by the equation

\begin{equation}
\rm K_{corr} = K \; \left(- \frac{a_0 + a_1 \; \mbox{\texttt{t}} + a_2 \; \mbox{\texttt{t}}^2}{\mbox{\texttt{R}}\;\mbox{\texttt{T}}} \; \mbox{\texttt{p}} + \frac{b_0 + b_1 \; \mbox{\texttt{t}} + b_2 \; \mbox{\texttt{t}}^2}{1000\;\mbox{\texttt{R}}\;\mbox{\texttt{T}}} \; 0.5 \; \mbox{\texttt{p}}^2\right)
\end{equation}

\noindent
Where $\rm K_{corr}$ is the pressure corrected constant and K is the uncorrected constant, both on matching units, e.g., mol/kg-soln, \texttt{T} is the absolute temperature in Kelvin, \texttt{t} is the temperature in \textdegree$ $C, \texttt{R} is the ideal gas constant in (bar $\rm cm^3$)/(mol Kelvin), and \texttt{p} is the gauge pressure (total pressure minus one atm, see \cite{Feistel2008} for a definition) in bar. The a and b coefficients (according to \citet{Millero1995} which is partly a restatement of \citet{Millero1979}, corrected by \citet{Lewis1998}) for constants in \textsf{AquaEnv} (stored in the data frame \texttt{DeltaPcoeffs}) are given in the following table\footnote{Note that in \citet{Lewis1998} it is stated that the $a$ values for $\rm H_2O$ and $\rm H_2S$ are \textit{freshwater} values! And that the coefficients for the silicate species are assumed to be the same as the ones for the borate species.}.\\

\noindent
Note that, while not stated in \citet{Millero1995}, it can be inferred from \cite{Lewis1998} and the code given by \cite{vanHeuven2009}, that the pressure correction is valid for \texttt{K\_HF} and \texttt{H\_H2SO4} on the free scale and for all other dissociation constants on the seawater pH scale. To be consistent with \cite{Lewis1998} and \cite{vanHeuven2009}, in \textsf{AquaEnv} all dissociation constants obtained from the original formulae are first converted to the free or seawater scale respectively using scale conversion factors with \texttt{K\_HF} and \texttt{H\_H2SO4} being not pressure corrected. Then the pressure correction is applied. Subsequently, the dissociation constants are converted to the desired pH scale with scale conversion factors with \texttt{K\_HF} and \texttt{H\_H2SO4} being pressure corrected.



\begin{center}
\begin{tabular}{|lrrrrrr|}\hline
              & $a_0$ & $a_1$ & $a_2$ & $b_0$ & $b_1$ & $b_2$\\ \hline
\texttt{K\_HSO4}        & -18.03& 0.0466& 0.3160 $10^{-3}$&- 4.53& 0.0900&0\\
\texttt{K\_HF}          &  -9.78&-0.0090&-0.9420 $10^{-3}$&- 3.91& 0.0540&0\\
\texttt{K\_CO2}         & -25.50& 0.1271& 0.0000 $10^{-3}$&- 3.08& 0.0877&0\\
\texttt{K\_HCO3}        & -15.82&-0.0219& 0.0000 $10^{-3}$&  1.13&-0.1475&0\\
\texttt{K\_W}           & -25.60& 0.2324&-3.6246 $10^{-3}$&- 5.13& 0.0794&0\\
\texttt{K\_BOH3}        & -29.48& 0.1622& 2.6080 $10^{-3}$&- 2.84& 0.0000&0\\
\texttt{K\_NH4}         & -26.43& 0.0889&-0.9050 $10^{-3}$&- 5.03& 0.0814&0\\
\texttt{K\_H2S}         & -14.80& 0.0020&-0.4000 $10^{-3}$&  2.89& 0.0540&0\\
\texttt{K\_H3PO4}       & -14.51& 0.1211&-0.3210 $10^{-3}$&- 2.67& 0.0427&0\\
\texttt{K\_H2PO4}       & -23.12& 0.1758&-2.6470 $10^{-3}$&- 5.15& 0.0900&0\\
\texttt{K\_HPO4}        & -26.57& 0.2020&-3.0420 $10^{-3}$&- 4.08& 0.0714&0\\
\texttt{K\_SiOH4}       & -29.48& 0.1622& 2.6080 $10^{-3}$&- 2.84& 0.0000&0\\
\texttt{K\_SiOOH3}      & -29.48& 0.1622& 2.6080 $10^{-3}$&- 2.84& 0.0000&0\\
\texttt{Ksp\_calcite}   & -48.76& 0.5304& 0.0000 $10^{-3}$&-11.76& 0.3692&0\\
\texttt{Ksp\_aragonite} & -45.96& 0.5304& 0.0000 $10^{-3}$&-11.76& 0.3692&0\\ \hline
\end{tabular}
\end{center}

\subsection{Conversion factors}
The following list gives a basic list of concentration and pH scale conversion factors used in \textsf{AquaEnv}. All other conversion factors, e.g., to be used in the function \texttt{convert}, are calculated from the factors given here. Note that the factors given below are multiplicative factors that can be used to convert e.g. dissociation constants or proton concentration values. To convert pH values, one needs to use the negative decadal logarithm of the factors below as an additive term. \texttt{molal2molin} signifies conversion from mol/kg-$\rm H_2O$ to mol/kg-soln, \texttt{free2tot} signifies conversion from the free to the total pH scale, \texttt{free2sws}  signifies conversion from the free to the seawater pH scale (for a general treatment of the free, total and seawater pH scale see \citet{Dickson1984} and \citet{Zeebe2001}), and \texttt{free2nbs} signifies conversion from the free to the NBS pH scale \citep{Durst1975}.
\begin{center}
\begin{tabular}{|llp{.6\textwidth}|} \hline
\texttt{molal2molin} & (1 - 0.001005 S) & \citet[p. 257]{Roy1993a}, \citet[chap. 5, p. 15]{DOE1994}\\
\texttt{free2tot} & $\rm (1 + \frac{S_T}{\mbox{\texttt{K\_HSO4}}})$ &\citet[p. 2302]{Dickson1984}, \citet[chap. 5, p. 16]{DOE1994}, \citet[p. 57, p. 261]{Zeebe2001}\\
\texttt{free2sws} &  $\rm (1 + \frac{S_T}{\mbox{\texttt{K\_HSO4}}} + \frac{F_T}{\mbox{\texttt{K\_HF}}})$ & \citet[p. 2303]{Dickson1984}, \citet{Zeebe2001}\\
\texttt{free2nbs} &  $\rm \gamma_{\rm H^+}$ / \texttt{molal2molin} & \citet{Dickson1984}, \citet{Lewis1998}, \citet{Zeebe2001}\\ \hline
\end{tabular}
\end{center}
In the above table \texttt{S} is salinity, $\rm S_T$ = $[\ce{SO_4^{2-}}] + [\ce{HSO_4^-}] \approx [\ce{SO_4^{2-}}]$, $\rm F_T$ = $\rm [\ce{HF}] + [\ce{F^-}] \approx [\ce{F^-}]$, both in mol/kg-soln, and $\rm \gamma_{\rm H^+}$ is the activity coefficient for the proton. The dissociation constants $\mbox{\texttt{K\_HSO4}}$ and $\mbox{\texttt{K\_HF}}$ are on the free pH scale and in mol/kg-soln as well. Note that, as given in \citet[p. 2303]{Dickson1984} and \citet[p. 91f]{Dickson1979} all concentrations appearing in the definition for the total and the seawater pH scale are molal, i.e.  mol/kg-$\rm H_2O$, concentrations. But in \citet[p. 257]{Roy1993a} and in \citet[chap.. 4, SOP 6, p. 1]{DOE1994} it is stated, that concentrations for the seawater and total pH scale are molin, i.e. mol/kg-soln. To be consistent with \citet{DOE1994} and \cite{Dickson2007} mol/kg-soln is chosen here for the free, total and seawater scale. To be consistent with \cite{Lewis1998}, the NBS scale is based on the proton concentration on mol/kg-H$_2$O.

\subsection{Activity coefficient for the proton}
In \textsf{AquaEnv} a complex ion-interaction model like, e.g., \citet{Millero1998} is not implemented.
According to \citet{Zeebe2001} the activity coefficient for the proton $\rm \gamma_{\rm H^+}$  can be approximated by the Davies equation as long as the ionic strength of the solution in question remains below 0.5 mol/kg-$\rm H_2O$. This means for solutions with a salinity of less than 24.48. Since NBS scale pH values are mostly not used for open ocean applications but mainly in brackish and fresh waters, the Davies equation has been assumed to be a sufficient approximation for $\rm \gamma_{\rm H^+}$. Important to note, however, is that \textbf{the conversion from and to the NBS pH scale in \textsf{AquaEnv} for salinities above 24.48 is only a rough approximation!}.
The Davies equation is used as given in \citet{Zeebe2001}
\begin{equation}
\rm \gamma_{\rm H^+}  =  10^{- \left(1.82 \; 10^6 \; \left(\epsilon \; T\right)^{-\frac{3}{2}}\right) \; \left( \frac{\sqrt(I)}{1 + \sqrt(I)} - 0.2 \; I\right)}
\end{equation}
where $\epsilon$ is the relative dielectric constant of seawater (\texttt{PhysChemConst\textdollar$ $e} in \textsf{AquaEnv}), \texttt{T} is the temperature in Kelvin, and \texttt{I} is the ionic strength in mol/kg-$\rm H_2O$. Note that the squared charge of the ion before the brackets with the ionic strength terms which is present in the generic form of the Davies equation has been omitted here since for the proton, this factor is 1.\\

\subsection{The revelle factor}
In \citet[p.73]{Zeebe2001} the revelle factor is given as
\begin{equation}
\mbox{\texttt{revelle}} = \frac{d [\rm CO_2]}{[\rm CO_2]} \Bigg/ \frac{d \rm [\sum CO_2]}{\rm [\sum CO_2]} \Bigg|_{[\rm TA] = \mbox{const.}}
\end{equation}
in \textsf{AquaEnv} \texttt{revelle} is calculated numerically.

\subsection{Partial derivatives of total alkalinity}
The values for \texttt{dTAdKdKdS}, \texttt{dTAdKdKdT}, \texttt{dTAdKdKdd}, \texttt{dTAdKdKdSumH2SO4}, and \texttt{dTAdKdKdSumHF} are calculated numerically as described in \citet{Hofmann2008b}.\\

\noindent
The values for \texttt{dTAdH},  \texttt{dTAdSumCO2}, \texttt{dTAdSumBOH3}, \texttt{dTAdSumH2SO4}, and  \texttt{dTAdSumHF}  are calculated analytically as given in \citet{Hofmann2008}.\\


\bibliography{AquaEnv}
\bibliographystyle{plainnat}


\end{document}
